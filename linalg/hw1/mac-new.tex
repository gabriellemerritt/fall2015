%%
%  macros to make a list of symbols index
%%
% \cases is replaced by \oldcases to avoid a conflict with amsmath
%  version dated 12/22/2010
%
\newbox\sample
\newif\ifproofmode
\newif\ifsymindex
\global\symindexfalse
\newwrite\inx
\def\indsyma#1#2{\ifproofmode\marginpar{$\scriptstyle#1$}\fi%
\ifx#2\empty\write\inx{$\noexpand#1$,\space\thepage}%
\write\inx{\string\newline}\else%
\write\inx{$\noexpand#1$,\space#2,\space\thepage}%
\write\inx{\string\newline}\fi\ignorespaces}%
%
\def\indsym#1#2{\ifsymindex%
\ifproofmode\marginpar{$\scriptstyle#1$}\fi%
\ifx#2\empty\write\inx{\string\item \space$\noexpand#1$,\space\thepage}%
\else%
\write\inx{\string\item \space$\noexpand#1$,\space#2,\space\thepage}%
\fi\ignorespaces\fi}%
%
\def\openifile#1{\ifsymindex\immediate\openout\inx=#1\fi}
\def\closeifile#1{\ifsymindex\immediate\closeout#1\fi}
%  open file for symbol index
%  and turn on printing symbol in the margin
\def\makesymt#1{\global\proofmodetrue\global\symindextrue\openifile{#1}}
%  open file for symbol index
%  but turn off printing symbol in the margin
\def\nomargs#1{\global\proofmodefalse\global\symindextrue\openifile{#1}}
%  index, but ignore spaces after item
\def\nsindex#1{\index{#1}\ignorespaces}

%%
% web page abbreviation
%%

\def\webref{\ http://www.cis.upenn.edu/$\widetilde{\;}$jean/gbooks/geom2.html}
\def\webpage{web page}%

%%
%  needed for danger sign
%%
\newskip\dangerskipb
\newskip\dangerskip
\dangerskip=20pt
\dangerskipb=42pt
\def\hang{\hangindent\dangerskip}
\def\hangb{\hangindent\dangerskipb}


\def\union{\cup}
\def\s#1{{\cal #1}}
\def\tensor{\otimes}
\def\lag{\left\langle}
\def\rag{\right\rangle}
\def\blag{\bigl\langle}
\def\brag{\bigr\rangle}
\def\lbr{\lbrack\!\lbrack}
\def\rbr{\rbrack\!\rbrack}

%\def\pairt#1#2{\lag #1,\; #2\rag}
\def\pairt#1#2{\lag #1, #2\rag}

\def\beginit{\begingroup\it}

\def\endit{\endgroup}

%\def\proof{\noindent{\it Proof\/}.\enspace}
%\def\proofs{\noindent{\it Proofs\/}.\enspace}
\def\hint{\noindent{\it Hint\/}.\enspace}
\def\anote{\noindent{\it Note\/}.\enspace}
\def\acase{\medskip\noindent{\it Case\/}\enspace}

%\def\endproof{$\bigsquare$}
%\def\endproof{\bigskip}

\def\remark{\bigskip\noindent{\bf Remark:}\enspace}
\def\endremark{\bigskip}
\def\remarks{\bigskip\noindent{\bf Remarks:}\enspace}
\def\endanger{\medskip}

% dangerous bend sign
\font\manual=manfnt at 12pt
%\font\specialfont=cmssmc40
%\font\specialfont=cmss17
\def\danbend{{\manual\char127}}
%
\def\datanger{\medbreak\begingroup\clubpenalty=10000
 \def\par{\endgraf\endgroup\medbreak} \noindent\hang\hangafter=-2
 \hbox to0pt{\hskip-3.5pc\danbend\hfill}}
\outer\def\danger{\datanger}%
%
\def\datangerb{\medbreak\begingroup\clubpenalty=10000
 \def\par{\endgraf\endgroup\medbreak} \noindent\hangb\hangafter=-2
 \hbox to0pt{\hskip-3.5pc\danbend\hfill}}
\outer\def\dangerb{\datangerb}
%
\def\datangerc{\medbreak\begingroup\clubpenalty=10000
 \def\par{\endgraf\endgroup\medbreak} \noindent\hang\hangafter=-3
 \hbox to0pt{\hskip-3.5pc\danbend\hfill}}
\outer\def\dangerc{\datangerc}
%
\def\datangerd{\medbreak\begingroup\clubpenalty=10000%
\def\par{\endgraf\endgroup\medbreak}\noindent\hang\hangafter=-4%
\hbox to0pt{\hskip-3.5pc\danbend\hfill}}%
\outer\def\dangerd{\datangerd}
%
\outer\def\dangerg#1{\medbreak\begingroup\clubpenalty=10000%
\def\par{\endgraf\endgroup\medbreak}\noindent\hang\hangafter=-#1%
\hbox to0pt{\hskip-3.5pc\danbend\hfill}}%

%
%
\def\datangerbis{\medbreak\begingroup\clubpenalty=10000%
\vbox to0pt{\hbox to0pt{\hskip-2.5pc\danbend\hfill}}
\endgraf\endgroup}
\outer\def\dangerbis{\datangerbis}
%
%
% double dangerous bend sign
%
\def\ddatanger{\medbreak\begingroup\clubpenalty=10000
 \def\par{\endgraf\endgroup\medbreak} \noindent\hang\hangafter=-2
 \hbox to0pt{\hskip-3.5pc\danbend\kern1pt%
\danbend\hfill}}
\outer\def\ddanger{\ddatanger}
\def\enddanger{\endgraf\endgroup}% omits the \medbreak
% For example, a \smallskip\item might be used to give an itemized list
% within the scope of the dangerous bend sign
%

\def\bull{\vrule height 1.7ex width 1.7ex depth -.1ex} 

\def\whyeq{\buildrel \rm ? \over =}

\def\forkmapp#1#2#3{ \succ\joinrel\rightarrow_{[#1,#2,#3]} }

\def\forkmappfour#1#2#3#4{ \succ\joinrel\rightarrow_{[#1,#2,#3,#4]} }

\def\bigjoinrelat{\mathrel{\mkern-26mu}}

\def\dobrightarrow{\longrightarrow\bigjoinrelat\longrightarrow}

%\def\dobrightarrow{\longrightarrow\!\!\!\!\!\longrightarrow}

\def\dobdownarrow{\mathop{\vbox{\kern2pt \hbox{$\Big\downarrow$}\kern-16.5pt
                          \nointerlineskip\hbox{$\Big\downarrow$}}}}

\def\maprightu#1{\smash{
   \mathop{\longrightarrow}\limits^{#1}}}

\def\mapleftu#1{\smash{
   \mathop{\longleftarrow}\limits^{#1}}}

\def\mapleftd#1{\smash{
   \mathop{\longleftarrow}\limits_{#1}}}

\def\maprightd#1{\smash{
   \mathop{\longrightarrow}\limits_{#1}}}

\def\mapdownr#1{\Big\downarrow
  \rlap{$\vcenter{\hbox{$\scriptstyle#1$}}$}}

\def\mapdownl#1{\llap{$\vcenter{\hbox{$\scriptstyle#1$}}$}
\Big\downarrow}

\def\mapupr#1{\Big\uparrow
  \rlap{$\vcenter{\hbox{$\scriptstyle#1$}}$}}

\def\mapupl#1{\llap{$\vcenter{\hbox{$\scriptstyle#1$}}$}
\Big\uparrow}

\def\maprightrewr#1#2#3{\longrightarrow_{[#1,#2,#3]}}

\def\maprightnarr#1#2#3{\forkmapp{#1}{#2}{#3}}

\def\maprightnarrfour#1#2#3#4{\forkmappfour{#1}{#2}{#3}{#4}}

% down, left to right, left label
\def\mapdiag#1{\llap{$\vcenter{\hbox{$\scriptstyle#1$}}$}
\searrow}

% down, right to left, right label
\def\mapantidiag#1{\rlap{$\vcenter{\hbox{$\quad\enspace\scriptstyle#1$}}$}
\swarrow}

%\def\mapdiagdr#1{\smash{\mathop{\searrow}\limits^{#1}}}

% down, left to right, left label
\def\mapdiagdr#1{\llap{$\vcenter{\hbox{$\scriptstyle#1$}}$}
\searrow}

% down, left to right, right label
\def\mapdiagdrr#1{\searrow\rlap{$\vcenter{\hbox{$\scriptstyle#1$}}$}}

% up, left to right, left label
\def\mapdiagul#1{\llap{$\vcenter{\hbox{$\scriptstyle#1$}}$}
\nearrow}

% up, left to right, right label
\def\mapdiagull#1{\nearrow\rlap{$\vcenter{\hbox{$\scriptstyle#1$}}$}}

% down, right to left, left label
\def\mapdiagdl#1{\llap{$\vcenter{\hbox{$\scriptstyle#1$}}$}
\swarrow}

% down, right to left, right label
\def\mapdiagdlr#1{\swarrow \rlap{$\vcenter{\hbox{$\scriptstyle#1$}}$}}

% up, right to left, left label
\def\mapdiaglu#1{\llap{$\vcenter{\hbox{$\scriptstyle#1$}}$}
\nwarrow}

% up, right to left, right label
\def\mapdiaglur#1{\nwarrow \rlap{$\vcenter{\hbox{$\scriptstyle#1$}}$}}

% up, left to right, left label
\def\mapdiagur#1{\rlap{$\vcenter{\hbox{$\scriptstyle#1$}}$}
\nearrow}

%  down, left to right, left label
\def\mapdiagdur#1{\rlap{$\vcenter{\hbox{$\scriptstyle#1$}}$}
\searrow}

%\def\mapdiagur#1{\smash{\mathop{\nearrow}\limits_{#1}}}

%, up, left to right, right label
\def\mapdiagurr#1{\nearrow\rlap{$\vcenter{\hbox{$\scriptstyle#1$}}$}}

%\def\mapdiagul#1{\smash{\mathop{\nearrow}\limits^{#1}}}

%\def\mapdiagdl#1{\smash{\mathop{\swarrow}\limits_{#1}}}







\def\mapadiagr#1{\swarrow \rlap{$\vcenter{\hbox{$\scriptstyle\>#1$}}$}}


\def\maprightudob#1{\smash{
   \mathop{\dobrightarrow}\limits^{#1}}}


\def\mapdownrdob#1{\dobdownarrow%
  \rlap{$\vcenter{\hbox{$\scriptstyle#1$}}$}}


\def\mapdownL#1{\llap{$\vcenter{\hbox{$\scriptstyle#1$}}$}
  \Big\downarrow}

\def\mapdownR#1{\Big\downarrow
  \rlap{$\vcenter{\hbox{$\scriptstyle#1$}}$}}


\def\lrightarrow{\hbox to 25pt{\rightarrowfill}}



\def\supexp{exp(m,n,p)=m^{m^{m^{\cdot^{\cdot^{\cdot^{m^{p}}}}}}}
\vbox{\hbox{$\Big\}\scriptstyle n$}\kern0pt}}

\def\supexpo#1#2#3{#1^{#1^{\cdot^{\cdot^{\cdot^{#1^{#2}}}}}}
\vbox{\hbox{$\Big\}\scriptstyle #3$}\kern0pt}}

\def\sqr#1#2{{\vcenter{\hrule height .#2pt
         \hbox{\vrule width.#2pt height#1pt \kern#1pt
             \vrule width.#2pt}
         \hrule height.#2pt}}}

\def\square{\mathchoice\sqr64\sqr64\sqr{2.1}3\sqr{1.5}3}

\def\bigsquare{\mathchoice\sqr76\sqr76\sqr{2.1}3\sqr{1.5}3}
\def\husquare{\mathchoice\sqr{16}{8}\sqr{16}{8}\sqr{2.1}3\sqr{1.5}3}

\def\lag{\langle}
\def\rag{\rangle}
\def\sequen{\triangleright}
\def\termdecl#1#2{#1\sequen #2}
\def\tdecomp#1#2{#1[#2]}
\def\co{\colon}


\def\rewrites#1{\ {\buildrel * \over \longleftrightarrow}_{#1}\ }

%
% These are needed to fake \displaylines, \eqalignno, \item 
% and \itemitem, when
% plain TEX would interfere (also \cases for LATEX)
%

% from page 347 of TEX manual
%
\newskip\bogcentering \bogcentering= 0pt plus 1000pt minus 1000pt 

%from page 353 of TEX manual
%
\def\matth{\mathsurround=0pt}
%
%from page 355 of TEX manual
%
\def\itemm{\par\hang\textindent}
\def\itemitemm{\par\indent \hangindent2\parindent \textindent}

%
%from page 357 of TEX manual
%
\def\fakrightarrowfill{$\matth \mathord- \mkern-6mu
  \cleaders\hbox{$\mkern-2mu \mathord- \mkern-2mu$}\hfill
 \mkern-6mu \mathord\rightarrow$}

%
%from page 359 of TEX manual
%
\def\fakoverrightarrow#1{\vbox{\ialign{##\crcr
  \fakrightarrowfill\crcr\noalign{\kern-1pt\nointerlineskip}
 $\hfil\displaystyle{#1}\hfil$\crcr}}}


%from page 362 of TEX manual
%
%\def\cases#1{\left\{\,\vcenter{\normalbaselines\matth
%  \ialign{$##\hfil$&\quad##\hfil\crcr#1\crcr}}\right.}

%
\def\oldcases#1{\left\{\,\vcenter{\normalbaselines\matth
  \ialign{$##\hfil$&\quad##\hfil\crcr#1\crcr}}\right.}

\newif\ifdtatp

\def\displaty{%
\global \dtatptrue \openup \jot \matth \everycr{\noalign{\ifdtatp \global 
\dtatpfalse \vskip -\lineskiplimit \vskip \normallineskiplimit \else 
\penalty \interdisplaylinepenalty \fi }}}

%
% Fake \displaylines
%
\def\displaylignes#1{\displaty
   \halign{\hbox to\displaywidth{$\displaystyle##$}\crcr
   #1\crcr}}
%
% Fake \eqalignno
%
\def\eqaligneno#1{\displaty \tabskip=\bogcentering
 \halign to\displaywidth{\hfil$\displaystyle{##}$\tabskip=0pt
 &$\displaystyle{{}##}$\hfil\tabskip=\bogcentering
 &\llap{$##$}\tabskip=0pt\crcr
 #1\crcr}}
%
% Fake \leqalignno
%
\def\leqaligneno#1{\displaty \tabskip=\bogcentering
 \halign to\displaywidth{\hfil$\displaystyle{##}$\tabskip=0pt
 &$\displaystyle{{}##}$\hfil\tabskip=\bogcentering
 &\kern-\displaywidth\rlap{$##$}\tabskip=\displaywidthpt\crcr
 #1\crcr}}
%
% Fake \line (from page 353 of TEX manual
%
\def\ligne{\hbox to\hsize}
%
% End of macros to fool LATEX
%



% macro to put a box around something (a \box)
\newdimen\nouvpagewidth
\newdimen\offwidth
\newdimen\lawidthoui
\lawidthoui=5.5truein
\offwidth=5truepc
\nouvpagewidth=\lawidthoui
%\nouvpagewidth=\hsize %\advance\newpagewidth by -\offwidth
\def\kboxit#1{\vbox{\hrule\hbox{\vrule\kern3pt
              \vbox{\kern3pt#1\kern3pt}\kern3pt\vrule}\hrule}}


% a single box around some text in #1, centered

\def\boxit#1{\setbox4=\vbox{\hsize \nouvpagewidth\noindent\strut#1\strut}
                 $$\kboxit{\box4}$$}

% a double box around some text in #1, centered

\def\dboxit#1{\setbox4=\vbox{\hsize \nouvpagewidth\noindent\strut#1\strut}
                 $$\kboxit{\kboxit{\box4}}$$}

% a single box around some text in #1

\def\boxaround#1{\setbox4=\hbox{\strut#1\strut}\kboxit{\box4}}

% special boxaround
\def\kboxitb#1{\vbox{\hrule\hbox{\vrule\kern3pt
              \vbox{\kern3pt#1\kern3pt}\kern3pt\vrule}\hrule}}

\def\laboxaround#1{
\aboxaround{\hbox to\hsize{\hfill\box2\hfill}}{#1}
}

\def\boxar#1#2{
\aboxaround{\hbox to\hsize{\hfill#1\hfill}}{#2}
}

\def\aboxaround#1#2{
\setbox4=\vbox{\hsize #2\noindent\strut#1\strut}
\kboxitb{\box4}}

% a double box around some text in #1

\def\dboxaround#1{\setbox4=\hbox{\strut#1\strut}\kboxit{\kboxit{\box4}}}


% macro to frame a box in order to make nice tables
\def\kframeit#1{\vbox{\hrule\hbox{\vrule\kern5pt
              \vbox{\kern5pt#1\kern5pt}\kern5pt\vrule}\hrule}}
% a single box around some \vbox in #1, centered
\def\frameit#1{\setbox4=#1$$\kframeit{\box4}$$}

% to frame a box, argument in math mode displaystyle
\def\laframe#1{\setbox4=\hbox{\strut{$\displaystyle #1$}\strut}%
\kframeit{\box4}}

% to double frame a box, argument in math mode displaystyle
\def\dlaframe#1{\setbox4=\hbox{\strut{$\displaystyle #1$}\strut}%
\kboxit{\kframeit{\box4}}}

\newskip\savnormalbaselineskip
\newskip\savnormallineskip
\newdimen\savnormallineskiplimit

\def\matrixx#1{
\savnormalbaselineskip=\normalbaselineskip
\savnormallineskip=\normallineskip
\savnormallineskiplimit=\normallineskiplimit
%
\normalbaselineskip=16pt
\normallineskip=3pt \normallineskiplimit3pt
%
\matrix{#1}
%
\normalbaselineskip=\savnormalbaselineskip
\normallineskip=\savnormallineskip
\normallineskiplimit=\savnormallineskiplimit}

% For use with LATEX
%
\def\matrice#1{
\savnormalbaselineskip=\normalbaselineskip
\savnormallineskip=\normallineskip
\savnormallineskiplimit=\normallineskiplimit
%
\normalbaselineskip=16pt
\normallineskip=2pt \normallineskiplimit2pt
%
\matrix{#1}
%
\normalbaselineskip=\savnormalbaselineskip
\normallineskip=\savnormallineskip
\normallineskiplimit=\savnormallineskiplimit}


\def\square{\bigsquare}









